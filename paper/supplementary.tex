%% For double-blind review submission, w/o CCS and ACM Reference (max submission space)
\documentclass[acmsmall,review,anonymous]{acmart}\settopmatter{printfolios=true,printccs=false,printacmref=false}
%% For double-blind review submission, w/ CCS and ACM Reference
%\documentclass[acmsmall,review,anonymous]{acmart}\settopmatter{printfolios=true}
%% For single-blind review submission, w/o CCS and ACM Reference (max submission space)
%\documentclass[acmsmall,review]{acmart}\settopmatter{printfolios=true,printccs=false,printacmref=false}
%% For single-blind review submission, w/ CCS and ACM Reference
%\documentclass[acmsmall,review]{acmart}\settopmatter{printfolios=true}
%% For final camera-ready submission, w/ required CCS and ACM Reference
%\documentclass[acmsmall]{acmart}\settopmatter{}


%% Journal information
%% Supplied to authors by publisher for camera-ready submission;
%% use defaults for review submission.
\acmJournal{PACMPL}
\acmVolume{1}
\acmNumber{CONF} % CONF = POPL or ICFP or OOPSLA
\acmArticle{1}
\acmYear{2018}
\acmMonth{1}
\acmDOI{} % \acmDOI{10.1145/nnnnnnn.nnnnnnn}
\startPage{1}

%% Copyright information
%% Supplied to authors (based on authors' rights management selection;
%% see authors.acm.org) by publisher for camera-ready submission;
%% use 'none' for review submission.
\setcopyright{none}
%\setcopyright{acmcopyright}
%\setcopyright{acmlicensed}
%\setcopyright{rightsretained}
%\copyrightyear{2018}           %% If different from \acmYear

%% Bibliography style
\bibliographystyle{ACM-Reference-Format}
%% Citation style
\citestyle{acmauthoryear}  %% For author/year citations
%\citestyle{acmnumeric}     %% For numeric citations
%\setcitestyle{nosort}      %% With 'acmnumeric', to disable automatic
                            %% sorting of references within a single citation;
                            %% e.g., \cite{Smith99,Carpenter05,Baker12}
                            %% rendered as [14,5,2] rather than [2,5,14].
%\setcitesyle{nocompress}   %% With 'acmnumeric', to disable automatic
                            %% compression of sequential references within a
                            %% single citation;
                            %% e.g., \cite{Baker12,Baker14,Baker16}
                            %% rendered as [2,3,4] rather than [2-4].


%%%%%%%%%%%%%%%%%%%%%%%%%%%%%%%%%%%%%%%%%%%%%%%%%%%%%%%%%%%%%%%%%%%%%%
%% Note: Authors migrating a paper from traditional SIGPLAN
%% proceedings format to PACMPL format must update the
%% '\documentclass' and topmatter commands above; see
%% 'acmart-pacmpl-template.tex'.
%%%%%%%%%%%%%%%%%%%%%%%%%%%%%%%%%%%%%%%%%%%%%%%%%%%%%%%%%%%%%%%%%%%%%%


%% Some recommended packages.
\usepackage{booktabs}   %% For formal tables:
                        %% http://ctan.org/pkg/booktabs
\usepackage{subcaption} %% For complex figures with subfigures/subcaptions
                        %% http://ctan.org/pkg/subcaption

\usepackage{tikz}
\usetikzlibrary{arrows,automata}
\usepackage{tikz-qtree} % Used for syntax trees
\usepackage{tikz-cd} % Used for commutative diagrams

\usepackage{syntax}

\usepackage{semantic}

\usepackage{marginnote}

\usepackage{listings}
\lstset{
  basicstyle=\ttfamily,
  basewidth={.5em,.5em},
}
\lstdefinelanguage{syncon}
  {morekeywords={syncon,infix,postfix,prefix,left,right,comment,precedence,forbid,except,type,builtin,rec,grouping,token},
   sensitive=true,
   morecomment=[l]{//},
   morecomment=[s]{/*}{*/},
   morecomment=[s][<.][.>],
   morestring=[b]",
  }
\newcommand{\ocaml}{\lstinline[language={[objective]caml}]}
\newcommand{\syncon}{\lstinline[language=syncon]}

%% symbol definitions used throughout the paper

\newcommand{\support}{\mathit{Supp}}
\newcommand{\NT}{V} % Set of nonterminals
\newcommand{\T}{\Sigma} % Set of terminals
\newcommand{\Labels}{L} % Set of labels
\newcommand{\yield}{\mathit{yield}} % yield of a parse tree
\newcommand{\semantic}{\mathit{unparen}} % remove semantically unimportant productions from a w' \in L(G')
\newcommand{\parse}{\mathit{parse}} % go from a w' \in L(G'_w) to a subset of L(G_t)
\newcommand{\words}{\mathit{words}} % go from a t \in L(G_t) to a subset of L(G'_w)
\newcommand{\alt}{\mathit{alt}} % go from a w \in L(G'_w) to a tuple of a basic word and a range bag.
\newcommand{\amb}{\mathit{amb}}

\newcommand{\basic}{\mathit{basic}} % go from a w \in L(G'_w) to the first element of \alt(w).
\newcommand{\rangebag}{\mathit{rangebag}} % go from a w \in L(G'_w) to the first element of \alt(w).
\newcommand{\rangeset}{\mathit{rangeset}} % go from a w \in L(G'_w) to the first element of \alt(w).
\newcommand{\altset}{\mathit{altset}} % go from a w \in L(G'_w) to \alt(w), except the second element is replaced by a set (that contains an element if the bag contained at least one of that element).
\newcommand{\lattice}{\mathit{lattice}} % go from a t \in L(G_t) to its lattice, partitioned by equality on \altset and ordered by subset on the rangeset.
\newcommand{\topw}{\top_w} % go from a tree to its top word
\newcommand{\botw}{\bot_w} % go from a tree to its bottom word
\newcommand{\localize}{\mathit{localize}} % go from a F \subseteq L(G_t) to a set of localized ambiguities, i.e., a set of subforests.
\newcommand{\range}[2]{#1\!-\!#2}
\newcommand{\regex}{\mathit{Reg}}
\newcommand{\reqpl}{\underline{(}}
\newcommand{\reqpr}{)}
\newcommand{\reqp}[1]{\reqpl#1\reqpr}
\newcommand{\pospl}{(}
\newcommand{\pospr}{)}
\newcommand{\posp}[1]{\pospl#1\pospr}
\newcommand{\dfa}{\mathit{dfa}} % Go from label to its corresponding dfa
\newcommand{\passthrough}{\mathit{passthrough}} % Go from label to a set of (label x bool), where each element denotes a child production that can be an only child (i.e., no other terminals whatsoever), and the bool states whether it requires parentheses
\newcommand{\labelnt}{\mathit{nt}} % Go from label to the non-terminal to which that labelled production belongs
\newcommand{\labelregex}{\mathit{regex}} % Go from label the regex of the production with that label

\synctex=1
\begin{document}
% TODO: Decide on British or American English.

% Submission may be up to 25 pages, excluding references

%% Title information
\title{Resolvable Ambiguity}         %% [Short Title] is optional;
                                        %% when present, will be used in
                                        %% header instead of Full Title.


%% Author information
%% Contents and number of authors suppressed with 'anonymous'.
%% Each author should be introduced by \author, followe by
%% \authornote (optional), \orcid (optional), \affiliation, and
%% \email.
%% An author may have multiple affiliations and/or emails; repeat the
%% appropriate command.
%% Many elements are not rendered, but should be provided for metadata
%% extraction tools.

%% Author with single affiliation.
\author{Viktor Palmkvist}
\authornote{with author1 note}          %% \authornote is optional;
                                        %% can be repeated if necessary
\orcid{nnnn-nnnn-nnnn-nnnn}             %% \orcid is optional
\affiliation{
  \position{Position1}
  \department{Department1}              %% \department is recommended
  \institution{KTH Royal Institute of Technology}            %% \institution is required
  \streetaddress{Street1 Address1}
  \city{Stockholm}
  \state{State1}
  \postcode{Post-Code1}
  \country{Sweden}                    %% \country is recommended
}
\email{vipa@kth.se}          %% \email is recommended

%% Author with two affiliations and emails.
\author{First2 Last2}
\authornote{with author2 note}          %% \authornote is optional;
                                        %% can be repeated if necessary
\orcid{nnnn-nnnn-nnnn-nnnn}             %% \orcid is optional
\affiliation{
  \position{Position2a}
  \department{Department2a}             %% \department is recommended
  \institution{Institution2a}           %% \institution is required
  \streetaddress{Street2a Address2a}
  \city{City2a}
  \state{State2a}
  \postcode{Post-Code2a}
  \country{Country2a}                   %% \country is recommended
}
\email{first2.last2@inst2a.com}         %% \email is recommended
\affiliation{
  \position{Position2b}
  \department{Department2b}             %% \department is recommended
  \institution{Institution2b}           %% \institution is required
  \streetaddress{Street3b Address2b}
  \city{City2b}
  \state{State2b}
  \postcode{Post-Code2b}
  \country{Country2b}                   %% \country is recommended
}
\email{first2.last2@inst2b.org}         %% \email is recommended


%% Abstract
%% Note: \begin{abstract}...\end{abstract} environment must come
%% before \maketitle command
\begin{abstract}
% Why?

Ever since the early 60s, the problem of deciding if a context-free
grammar is unambiguous or not has been known to be undecidable. As a
consequence, a large number of restricted grammars have been developed
to guarantee that a language definition is unambiguous. This
traditional view of only allowing unambiguous grammars has until
recently been taken for granted as the only truth: \emph{the} way of
how the syntax of a language must be defined. However, up front
unambiguous grammars can force language designers to make arbitrary
choice to disambiguate the language, although the natural choice may be to
postpone the decision to the programmer. Moreover, an important
objective of domain-specific language development is to construct
languages by composing and extending existing
languages. Unfortunately, the composition of restricted unambiguous
grammars easily results in ambiguous grammars, which again are
undecidable. In this paper, we depart from the traditional view of
unambiguous grammar design, and allow ambiguities to be delayed until
compile time, allowing the user to perform the disambiguation. A
natural decision problems follows: given a language definition, can a
user always disambiguate an ambiguous program?  We introduce and
formalize this fundamental problem---called the \emph{resolvable
  ambiguity problem}---and divide it into separate static and dynamic
resolvability problems. We prove soundness and completeness for a
restricted language in the static case, and soundness in the general
case for dynamic resolvability. The approach is evaluated through
three separate case studies, covering both a large
existing programming language, and the composability of
domain-specific languages.
%A traditional view when designing programming languages is that syntax
%definitions should be unambiguous: only unambiguous grammars are used
%to construct language specific parsers that either rejects a program,
%or accepts it and produces a unique parse tree. However,
\end{abstract}


%% 2012 ACM Computing Classification System (CSS) concepts
%% Generate at 'http://dl.acm.org/ccs/ccs.cfm'.
\begin{CCSXML}
<ccs2012>
<concept>
<concept_id>10011007.10011006.10011008</concept_id>
<concept_desc>Software and its engineering~General programming languages</concept_desc>
<concept_significance>500</concept_significance>
</concept>
<concept>
<concept_id>10003456.10003457.10003521.10003525</concept_id>
<concept_desc>Social and professional topics~History of programming languages</concept_desc>
<concept_significance>300</concept_significance>
</concept>
</ccs2012>
\end{CCSXML}

\ccsdesc[500]{Software and its engineering~General programming languages}
\ccsdesc[300]{Social and professional topics~History of programming languages}
%% End of generated code


%% Keywords
%% comma separated list
\keywords{Language Composition, Context Free Grammars, Domain-Specific Languages}  %% \keywords are mandatory in final camera-ready submission


%% \maketitle
%% Note: \maketitle command must come after title commands, author
%% commands, abstract environment, Computing Classification System
%% environment and commands, and keywords command.
\maketitle

\section{Static Proofs}

\begin{lemma}
  There is a bijection between successful runs in $A_{\posp{}}$ and trees $t \in L(T_D)$.
\end{lemma}

\begin{proof}
  $A_{\posp{}}$ has a single initial state, and a single final state, call them $s$ and $f$ respectively. The two are distinct, and $f$ is reachable only through transitions that pop a stack symbol with $s$ and $f$ as the two first components. Transitions that push such symbols originate in $s$, and no other transitions originate in $s$. Thus all successful runs start in configuration $(s, \epsilon)$ and end in $(f, \epsilon)$, whereby every pushing transition in the run can be paired with a corresponding popping transition. We will now construct a tree $t \in L(T_D)$ in a bottom-up fashion.

  Select a subsequence of the run $R = (s, \epsilon) \; R_1 \; (p, s) \xrightarrow{a, +(p, q, w)} (p', (p, q, w) \cdot s) \; R_2 \; (q', (p, q, w) \cdot s \xrightarrow{b, -(p, q, w)} (q, s) \; R_2 \; (f, \epsilon)$ such that $R_2$ contains no pushing or popping operations. The only transitions that leave the stack unchanged in $A_{\posp{}}$ come from DFAs generated from productions in $D$. There is no way to transition between two such DFAs without using a pushing or popping transition, thus $R_2$ must be a successful run in the DFA for some labelled production $N -> l : r$ in $D$. $T_D$ must contain a corresponding production $N -> l(r)$. Now consider the sequence of input symbols $w'$ along the transitions in $R_2$. We have $w' \in L(r)$, since $R_2$ is a successful run in a DFA constructed from $r$. Thus $l(w') \in L_N(T_D)$. Now add parent nodes to this subtree for every label in $w$, e.g., if $w = l_1 \cdot l_2 \cdot l_3$ construct $l_1(l_2(l_3(l(w'))))$. We can then replace the subsequence of $R$ with this subtree, and then repeat the process until we have a tree with root $l'$, where $D$ has a labelled production $S -> l': r'$ for some $r'$ and the starting symbol $S$.

  To construct a successful run from a tree $t \in L(T_D)$ we can traverse the tree in a preorder, while at the same time constructing a run in $A_{\posp{}}$ starting from $(s, \epsilon)$. If the next terminal is in $\T$, i.e., a strictly zero-arity terminal, we take a transition originating in a DFA, which must be unique. If the next terminal is in $\Labels$, i.e., an unranked terminal, we take one of the added edges. However, this choice of edge might not be unique: there may be multiple edges from the current state to an appropriate DFA initial state. These stem from the original DFA having two transitions from the same state with the same label but different subscripts, e.g., $p \xrightarrow{l_\bot} q$ and $p \xrightarrow{l_\top} q'$. However, if both of these transitions were valid choices for the current tree that would imply that the DFA for that level can recognize two distinct words that differ only in subscripts. But distinctly subscripted terminals must originate from distinct occurances of marked non-terminals (since a $\top$ subscript corresponds to a mark, and a $\bot$ subscript corresponds to the absence of a mark), which would imply that the original regular expression is ambiguous if the marks are removed, which we explicitly ruled out in Section~\ref{sec:parse-time-disambiguation}.

% TODO: technically, we should perhaps mention that these two constructions are inverses of each other, but that's relatively easy to see once you understand them.
\end{proof}

%% Bibliography
\bibliography{All}


%% %% Appendix
%% \appendix
%% \section{Appendix}

%% Text of appendix \ldots

\end{document}
